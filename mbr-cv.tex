%%%%%%%%%%%%%%%%%%%%%%%%%%%%%%%%%%%%%%%%%
% Freeman Curriculum Vitae
% XeLaTeX Template
% Version 2.0 (19/3/2018)
%
% This template originates from:
% http://www.LaTeXTemplates.com
%
% Authors:
% Vel (vel@LaTeXTemplates.com)
% Alessandro Plasmati
%
% License:
% CC BY-NC-SA 3.0 (http://creativecommons.org/licenses/by-nc-sa/3.0/)
%
%!TEX program = xelatex
% NOTICE: This template must be compiled with XeLaTeX, the line above should
% ensure this happens automatically but if it doesn't you will need to specify
% XeLaTeX as the engine in your editor or script
%
%%%%%%%%%%%%%%%%%%%%%%%%%%%%%%%%%%%%%%%%%

%----------------------------------------------------------------------------------------
%	PACKAGES AND OTHER DOCUMENT CONFIGURATIONS
%----------------------------------------------------------------------------------------

\documentclass[10pt]{article} % Font size, can be: 10pt, 11pt or 12pt

\input{structure.tex} % Include the file that specifies the document structure

% Headers and footers can be added with the \lhead{} \rhead{} \lfoot{} \rfoot{} commands
% Example right footer:
%\rfoot{\color{headings}{\sffamily Last update: \today. Typeset with Xe\LaTeX}}
\usepackage{enumitem}

%----------------------------------------------------------------------------------------

\begin{document}

\begin{paracol}{2} % Begin the multi-column environment

%----------------------------------------------------------------------------------------
%	NAME AND CURRICULUM VITAE TEXT
%----------------------------------------------------------------------------------------

\parbox[top][0.12\textheight][c]{\linewidth}{ % Parbox to hold the author name and CV text; fixed height to match the coloured box to the right, centred vertically and full line width
	\vspace{-0.04\textheight} % Reduce whitespace above the parbox to separate it from the main content
	\centering % Centre text
	{\sffamily\Huge Matthew Rossetter}\\\medskip % Your name
	{\Huge\color{headings}\cvtextfont Curriculum Vitae}
}
%----------------------------------------------------------------------------------------
%	EDUCATION
%----------------------------------------------------------------------------------------

\section{Education}

% Blank \educationentry{} command to add another degree:

%\educationentry{} % Duration
%{} % Degree
%{} % Honours, achievements or distinctions (e.g. first class honours)
%{} % Department
%{} % Institution

% All 5 parameters must be supplied but any can be empty if you don't need them

%------------------------------------------------

\begin{supertabular}{rp{6.7cm}} % Start a table with two columns, the table will ensure everything is aligned

	%------------------------------------------------

	\educationentry{2016 -- present} % Duration
	{Master of Physics} % Degree
    {Statement of Marks:\setlist{nolistsep}
        \begin{itemize}[noitemsep]
        \item Third Year - 2:1 (69\%)
        \item Second Year - 1st (77\%)
        \item First Year - 1st (72\%)
    \end{itemize}\vspace{-10pt}} % Honours, achievements or distinctions (e.g. first class honours)
	{Theoretical Physics} % Department
	{University of Durham} % Institution

	%------------------------------------------------

	\educationentry{2009 -- 2016} % Duration
	{Secondary Education} % Degree
	{} % Honours, achievements or distinctions (e.g. first class honours)
    {\vspace{-5pt}\setlist{nolistsep}
     \begin{itemize}[noitemsep]
        \item Advanced Highers - A in Physics, Maths, Chemistry, and English
        \item Highers - A in Physics, Chemistry, Biology, Maths, and English
        \item National $5$s - $7$ As and $1$ C including Maths, Physics, and English at A
     \end{itemize}\vspace{-10pt}} % Department
     {The High School of Glasgow} % Institution

	%------------------------------------------------

\end{supertabular}


%----------------------------------------------------------------------------------------
%	MAJOR RESEARCH PROJECT
%----------------------------------------------------------------------------------------

\section{Masters Thesis}

{\raggedright\textbf{``CP Violation In and Beyond The Standard Model: Two Higgs Doublet Model Type II Contributions to Flavour Observables"}\\\medskip}

In this study, we first cover the theory of the Standard Model (SM) and then test the Two Higgs Doublet Model (2HDM) of Type II as an extension to the SM using indicative flavour observables, such as leptonic and semileptonic decays of $B$ and $D$ mesons, $B\bar{B}$ mixing, and the $b\to s\gamma$ radiative decay.
Testing the 2HDM parameter space ($m_{H^+},\tan\beta$) to find alignment between theoretical calculations and experiment, constraints on the parameters were found for flavour phenomena to work towards a global fit.
We perform this fit in both the alignment and wrong sign limits of the 2HDM.
Strongly dominated by the $b\to s\gamma$ branching ratio, the mass of a charged Higgs particle would be expected to have a minimum value at 95\% CL of $490$ GeV in the alignment limit and $740$ GeV in the wrong sign limit.
The value of $\tan\beta$ is limited by $B_q\to\mu^+\mu^-$ decays, yielding maximum values at 95\% CL of $20.8$ in the alignment limit and $4.03$ in the wrong sign, for the fixed choices of other parameters.
This work was then joined with constraints from the oblique parameters $S,T,U$ from another work, to better give judgement on the state of the 2HDM fit.
The results of this combined fit are no more conclusive than flavour alone, due to additional parameter dependency in $S,T,U$; the only additional constraint we definitively find here is that $m_{H^+}\approx m_{H^0}\approx m_{A^0}$ in both the alignment and wrong sign limits.
The validity of this fit is heavily influenced by the semileptonic ratio $\mathcal{R}(D^*)$ which remains in disagreement with the 2HDM fit to $3\sigma$; the statistical fit of these observables points to exclusion of this model at 95\% CL in the wrong sign limit, and 85\% CL in the alignment limit, at $2\sigma$ error significance.
This model cannot be excluded at $3\sigma$ error significance, where $\mathcal{R}(D^*)$ no longer causes disagreement in the fit.

%\medskip % Extra whitespace before the next section

%------------------------------------------------

\vspace{-\baselineskip}\medskip % Standardise the whitespace after this section and before the next (the custom command adds too much otherwise)

%----------------------------------------------------------------------------------------
%	PUBLICATIONS
%----------------------------------------------------------------------------------------

\section{Academic Experiences}
%
%% Example \longformdescription{} command to add another publication:
%
%%\longformpublication{Reference (format this manually as desired)}
%
%%------------------------------------------------
%
\longformdescription{Quarkonia Modelling}{Throughout my degree course, I have had several courseworks that have required extensive Python programming, thus far leading up to the year-long project in my Third Year, where I modelled the early (n,l) wavefunctions for a Hydrogen atom, and then applied this to Quarkonia phenomena. I was able to calculate good approximations for several lower state masses, with the inclusion of hyperfine splitting. A full repository of this study can be found in my \href{https://github.com/mbr-phys/}{github profile.}}

\longformdescription{Python Tutoring}{As part of my fourth year, I have volunteered to help as a tutor in First Year programming workshops, assisting in teaching the new students the fundamentals of Python programming for physics. I put my knowledge of Python to good use in sharing it with newer students, and this has allowed me to gain a better understanding of the language in teaching it.}

%\longformdescription{\LaTeX Lecture Notes}{Throughout my undergraduate degree, I have created sets of lecture notes for most of my attended courses, helping my lecturers keep track  }
%\longformpublication{\textbf{Freeman, G. R.} (1996). Chemistry of Multiply Charged Negative Molecular Ions and Clusters in the Gas Phase:  Terrestrial and in Intense Galactic Magnetic Fields. \textit{The Journal of Physical Chemistry}, \textit{100}(11), 4331-4338.}
%
%\longformpublication{Jacobsen, F. M., Gee, N., \textbf{Freeman, G. R.} (1986). Electron mobility in liquid krypton as function of density, temperature, and electric field strength. \textit{Physical Review A}, \textit{34}(3): 2329-2335.}
%
%%------------------------------------------------
%
%% As an alternative to a long-form publication list, you can create a shorter summary using only DOI values and years.
%
%% Example \doipublication{} command to add another publication:
%
%%\doipublication{Year}{DOI}{firstauthor}{spaceafter}
%
%% All four parameters are required (can be empty though)
%% A value of "firstauthor" in the third parameter will print the DOI in bold
%% A "spaceafter" value in the fourth parameter will add some vertical space -- this is to be used between years
%
%%------------------------------------------------
%
%\begin{supertabular}{rl} % Start a table with two columns, the table will ensure everything is aligned
%
%	%------------------------------------------------
%
%	\doipublication{1996}{10.1021/jp951483+}{firstauthor}{spaceafter}
%
%	%------------------------------------------------
%
%	\doipublication{1990}{10.1139/p90-097}{firstauthor}{spaceafter}
%	\doipublication{1986}{10.1139/v86-297}{}{}
%
%	%------------------------------------------------
%
%	\doipublication{1986}{10.1103/PhysRevA.34.2329}{}{spaceafter}
%
%	%------------------------------------------------
%
%	& \textit{First author publications in} \textbf{bold}\\
%
%	%------------------------------------------------
%
%\end{supertabular}

%\medskip % Extra whitespace before the next section

%\medskip % Extra whitespace before the next section

%----------------------------------------------------------------------------------------
%	WORK EXPERIENCE
%----------------------------------------------------------------------------------------

\section{Work Experience}

% Blank \workposition command to add another job:

%\workposition{} % Duration
%{} % FT/PT (full time or part time)
%{} % Employer
%{} % Job title
%{} % Description

% All 5 parameters must be supplied but any can be empty if you don't need them

%------------------------------------------------

\workposition{Current, from Oct 2018} % Duration
{PT} % FT/PT (full time or part time)
{Freemans Quay Leisure Centre} % Employer
{Casual Leisure Assistant/Lifeguard} % Job title
{The challenge of being a lifeguard is in the potential for life-or-death situations. As a lifeguard, I present a positive leisure experience, which focuses on dealing with any customer issues and maintaining an enjoyable environment for all. This must be balanced with constant vigilance over the pool area in case of the need for intervention or rescue, with the proper knowledge and ability to save any lives that may be at risk.} % Description

%------------------------------------------------

\vspace{-10pt}
\workposition{Current, from Oct 2017} % Duration
{PT} % FT/PT (full time or part time)
{St Mary's College Bar} % Employer
{Bar Staff} % Job title
{I am responsible for serving and interacting with customers. This requires developing keen social skills to balance a friendly atmosphere for customers with a presence of authority to those who may cause trouble. It is important to maintain a high level of cleanliness in a frequently messy environment, and keep a mind for potential welfare issues such as binge drinking which must be monitored and reported.}  % Description

%------------------------------------------------

\workposition{Sep 2018 -- July 2019} % Duration
{PT} % FT/PT (full time or part time)
{St Mary's College Bar} % Employer
{Bar Steward} % Job title
{Over the academic year 18/19, I was in charge of the college bar with duties such as stock control, staff hiring and training, shift assignment, and maintainence and cleaning. I was selected for my strong work ethic and commitment to the role, and developed skills into this management role to improve the college bar to bring in more customers, creating a safer and more efficient working environment for staff. When managing staff who were also my peers, it was important to find the proper balance in the workplace so I would be taken seriously in my role as their manager.} % Description


%----------------------------------------------------------------------------------------

\switchcolumn % Switch to the next paracol column

%----------------------------------------------------------------------------------------
%	COLOURED CONTACT DETAILS BOX
%----------------------------------------------------------------------------------------

\parbox[top][0.12\textheight][c]{\linewidth}{ % Parbox to hold the colour box; fixed height to match the name/CV text to the left, centred vertically and full line width
	\vspace{-0.04\textheight} % Reduce whitespace above the parbox to separate it from the main content
	\colorbox{shade}{ % Create the coloured box
		\begin{supertabular}{p{0.05\linewidth}|p{0.775\linewidth}} % Start a table with two columns, the table will ensure everything is aligned
			\raisebox{-1pt}{\faHome} & Easter Banknock, Denny, FK6 5NA \\ % Address
			\raisebox{-1pt}{\faPhone} & +44 7446 949025, DoB: 26/10/1998 \\ % Phone number
            \raisebox{0pt}{\small\faEnvelope} & \href{mailto:mbr-phys@protonmail.com}{mbr-phys@protonmail.com}, \href{mailto:matthew.rossetter@durham.ac.uk}{matthew.rossetter@durham.ac.uk} \\ % Email address
			%\raisebox{-1pt}{\small\faDesktop} & \href{https://www.LaTeXTemplates.com}{https://www.LaTeXTemplates.com} \\ % Website
			\raisebox{-1pt}{\faGithub} & \href{https://github.com/mbr-phys}{https://github.com/mbr-phys} \\ % GitHub profile
			%\raisebox{-1pt}{\faLinkedinSquare} & \href{https://www.linkedin.com/in/username}{https://www.linkedin.com/in/username} \\ % LinkedIn profile
			% See fontawesome.pdf in the fonts folder for all icons you can use
		\end{supertabular}
	}
}

%----------------------------------------------------------------------------------------
%	AWARDS
%----------------------------------------------------------------------------------------

\section{Awards}

% Example \tableentry{} command to add another line:

%\tableentry{Heading}{Content}{spaceafter}

% All 3 parameters must be supplied but any can be empty if you don't need them
% A "spaceafter" value in the third parameter will add some vertical space -- this is to be used between headings

%------------------------------------------------

\begin{supertabular}{rp{6cm}} % Start a table with two columns, the table will ensure everything is aligned

	%------------------------------------------------
    \tableentry{2019}{President's Vote of Thanks, and Full Colours}{}
    \tableentry{}{\textit{St Mary's College, Durham University}}{spaceafter}

    \tableentry{2018}{BIIAB Level $2$ Award for Personal License Holders}{}
	\tableentry{2017}{RLSS National Pool Lifeguard Qualification}{}
    \tableentry{2016}{British Red Cross Emergency First Aid and ITC Outdoor First Aid}{}
	\tableentry{2015,16}{National Finalist and Company President at UK Space Design Competition}{}
    \tableentry{2014,15}{CREST Bronze \& Silver Science Awards}{}

	%------------------------------------------------

\end{supertabular}

%----------------------------------------------------------------------------------------
%	COMPUTER SKILLS
%----------------------------------------------------------------------------------------

\section{Computer Skills}

% Example \tableentry{} command to add another line:

%\tableentry{Heading}{Content}{spaceafter}

% All 3 parameters must be supplied but any can be empty if you don't need them
% A "spaceafter" value in the third parameter will add some vertical space -- this is to be used between headings

%------------------------------------------------

\begin{supertabular}{rl} % Start a table with two columns, the table will ensure everything is aligned

	%------------------------------------------------

	\tableentry{Beginner}{Perl, HTML, Fortran}{spaceafter}

	%------------------------------------------------

    \tableentry{Intermediate}{C++, Microsoft Office}{spaceafter}

	%------------------------------------------------

	\tableentry{Expert}{Python, Unix, \LaTeX}{spaceafter}

	%------------------------------------------------

\end{supertabular}


%----------------------------------------------------------------------------------------
%	COMMUNICATION SKILLS
%----------------------------------------------------------------------------------------

%\section{Communication Skills}
%
%% Example \tableentry{} command to add another line:
%
%%\tableentry{Heading}{Content}{spaceafter}
%
%% All 3 parameters must be supplied but any can be empty if you don't need them
%% A "spaceafter" value in the third parameter will add some vertical space -- this is to be used between headings
%
%%------------------------------------------------
%
%\begin{supertabular}{rl} % Start a table with two columns, the table will ensure everything is aligned
%
%	%------------------------------------------------
%
%	\tableentry{Conferences}{Oral Presentation at the Annual MIT}{}
%	\tableentry{}{Theoretical Physics Conference -- 1987}{spaceafter}
%
%	%------------------------------------------------
%
%	\tableentry{Posters}{Poster at the Meeting of the American}{}
%	\tableentry{}{Physical Society -- 1985}{spaceafter}
%
%	%------------------------------------------------
%
%\end{supertabular}

%----------------------------------------------------------------------------------------
%	SKILLS DESCRIPTION
%----------------------------------------------------------------------------------------

\section{Skills}

% Example \longformdescription{} command to add another section:

%\longformdescription{Heading}{Description}

%------------------------------------------------

\longformdescription{Critical Thinking}{I am able to conceptualise and analyse complex ideas and concepts. It is an important part of any role, particularly in science, to perceive the merits and flaws of a system and comprehend how to optimise this based on the information provided. One must also know when a system is flawed and must be reviewed.}

\longformdescription{Goal Oriented and Dedicated}{Understanding tasks properly is important to me. Performing these tasks properly is important to the efficient running of a workspace. I use my judgement to act and achieve goals quickly and efficiently. As part of a workplace, I dedicate myself fully to the task at hand. Focusing on what task needs done and putting the full effort needed towards its completion results in a better job being performed.}

\longformdescription{Passionate}{I have been interested in theoretical physics such as quantum mechanics and particle phenomenology from an early age. My education and research have cemented this interest into a passion. I greatly enjoy carrying out fundamental physics research with potential experimental implications.}

\longformdescription{Communication Skills and Collaborative Working}{I have spent a lot of time working in customer service and enjoy interacting with people of all backgrounds. I have learned how to work together with other people to achieve a goal, and adapt to many working environments to best make use of my resources.}

%\longformdescription{Manual Handling}{I have recieved manual handling training from several of my previous employers and am comfortable performing tasks that require this. I am comfortable switching from more inactive jobs and tasks to highly active ones.}

%----------------------------------------------------------------------------------------
%----------------------------------------------------------------------------------------
%	REFERENCES
%----------------------------------------------------------------------------------------

\section{References}

%\textit{References available on request}

%------------------------------------------------

% Example \tableentry{} command to add another line:

%\tableentry{Heading}{Content}{spaceafter}

% All 3 parameters must be supplied but any can be empty if you don't need them
% A "spaceafter" value in the third parameter will add some vertical space -- this is to be used between headings

%------------------------------------------------

\begin{supertabular}{rl} % Start a table with two columns, the table will ensure everything is aligned

	%------------------------------------------------

	\tableentry{}{\textbf{Dr. Alexander Lenz}}{spaceafter}
	\tableentry{Position}{Professor}{}
    \tableentry{Employer}{\href{https://www.ippp.dur.ac.uk}{\textit{Institute for Particle Physics Phenomenology}}}{}
    \tableentry{}{\href{https://www.dur.ac.uk}{\textit{University of Durham}}}{spaceafter}
	\tableentry{Email}{\href{mailto:alexander.lenz@durham.ac.uk}{alexander.lenz@durham.ac.uk}}{}
    \tableentry{Phone}{+44 (0)191 3343814}{}
%	\tableentry{Mobile}{+1 (201) 632-3901}{}

    %------------------------------------------------

	\tableentry{}{}{} % Creates some additional whitespace between the references

	%------------------------------------------------

    \tableentry{}{\textbf{Dr. Qing (Helen) He}}{spaceafter}
    \tableentry{Position}{Assistant Professor}{}
    \tableentry{Employer}{\href{https://www.dur.ac.uk/cmp}{\textit{Centre for Materials Physics}}}{}
    \tableentry{}{\href{https://www.dur.ac.uk}{\textit{University of Durham}}}{spaceafter}
	\tableentry{Email}{\href{mailto:qing.he@durham.ac.uk}{qing.he@durham.ac.uk}}{}
    \tableentry{Phone}{+44 (0)191 3343812}{}

	%------------------------------------------------

	\tableentry{}{}{} % Creates some additional whitespace between the references

	%------------------------------------------------
	\tableentry{}{\textbf{Mustafa Gun}}{spaceafter}
	\tableentry{Position}{Food and Beverages Services Manager}{}
    \tableentry{Employer}{\textit{St Mary's College}}{}
	\tableentry{}{\href{https://www.dur.ac.uk}{\textit{University of Durham}}}{spaceafter}
    \tableentry{Email}{\href{mailto:mustafa.gun@durham.ac.uk}{mustafa.gun@durham.ac.uk}}{}
    \tableentry{Phone}{+44 (0)191 3345920}{}
%	\tableentry{Mobile}{+1 (232) 842-3583}{}


	%------------------------------------------------

\end{supertabular}



\end{paracol}

%----------------------------------------------------------------------------------------

\end{document}
