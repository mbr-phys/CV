%!TEX program = xelatex
%!TEX encoding = UTF-8 Unicode
% Awesome CV LaTeX Template for Cover Letter
%
% This template has been downloaded from:
% https://github.com/posquit0/Awesome-CV
%
% Authors:
% Claud D. Park <posquit0.bj@gmail.com>
% Lars Richter <mail@ayeks.de>
%
% Template license:
% CC BY-SA 4.0 (https://creativecommons.org/licenses/by-sa/4.0/)
%


%-------------------------------------------------------------------------------
% CONFIGURATIONS
%-------------------------------------------------------------------------------
% A4 paper size by default, use 'letterpaper' for US letter
\documentclass[11pt, a4paper]{awesome-cv}

% Configure page margins with geometry
\geometry{left=1.4cm, top=.8cm, right=1.4cm, bottom=1.8cm, footskip=.5cm}
\newcommand\research{Particle Physics Group}
\newcommand\country{Seattle, Washington, USA}
\newcommand\uni{University of Washington}

% Specify the location of the included fonts
\fontdir[fonts/]

% Color for highlights
% Awesome Colors: awesome-emerald, awesome-skyblue, awesome-red, awesome-pink, awesome-orange
%                 awesome-nephritis, awesome-concrete, awesome-darknight
\colorlet{awesome}{awesome-red}
\def\red{\color{awesome}}
% Uncomment if you would like to specify your own color
% \definecolor{awesome}{HTML}{CA63A8}

% Colors for text
% Uncomment if you would like to specify your own color
% \definecolor{darktext}{HTML}{414141}
% \definecolor{text}{HTML}{333333}
% \definecolor{graytext}{HTML}{5D5D5D}
% \definecolor{lighttext}{HTML}{999999}

% Set false if you don't want to highlight section with awesome color
\setbool{acvSectionColorHighlight}{false}

% If you would like to change the social information separator from a pipe (|) to something else
\renewcommand{\acvHeaderSocialSep}{\quad\textbar\quad}


%-------------------------------------------------------------------------------
%	PERSONAL INFORMATION
%	Comment any of the lines below if they are not required
%-------------------------------------------------------------------------------
% Available options: circle|rectangle,edge/noedge,left/right
%\photo[circle,noedge,left]{profile}
\name{Matthew}{Black}
%\position{Software Architect{\enskip\cdotp\enskip}Security Expert}
\address{K\"olner Str. 17, 57072 Siegen, Germany}

\mobile{(+49) 173 917 3782}
\github{mbr-phys}
\email{matthewkblack@protonmail.com}
% \gitlab{gitlab-id}
% \stackoverflow{SO-id}{SO-name}
% \reddit{reddit-id}
% \medium{madium-id}
% \googlescholar{googlescholar-id}{name-to-display}
%% \firstname and \lastname will be used
% \googlescholar{googlescholar-id}{}
% \extrainfo{extra informations}

%-------------------------------------------------------------------------------
%	LETTER INFORMATION
%	All of the below lines must be filled out
%-------------------------------------------------------------------------------
% The company being applied to
\recipient
  {\research}
  {\uni \\ \country}
% The date on the letter, default is the date of compilation
\letterdate{\today}
% The title of the letter
\lettertitle{}
% How the letter is opened
\letteropening{}
% How the letter is closed
\letterclosing{Thank you for your time and consideration, I look forward to hearing from you. \\ Sincerely,}
% Any enclosures with the letter
%\letterenclosure[Attached]{Curriculum Vitae, Academic Transcript, Masters Thesis}


%-------------------------------------------------------------------------------
\begin{document}

% Print the header with above personal informations
% Give optional argument to change alignment(C: center, L: left, R: right)
\makecvheader[R]

% Print the footer with 3 arguments(<left>, <center>, <right>)
% Leave any of these blank if they are not needed
\makecvfooter
  {\today}
  {Matthew Black~~~·~~~Statement of Research}
  {}

% Print the title with above letter informations
%\makelettertitle

%-------------------------------------------------------------------------------
%	LETTER CONTENT
%-------------------------------------------------------------------------------
\begin{cvletter}

\lettersection{Statement of Research}
%\begin{itemize}
%    \item Masters project
%    \item 2HDM -> BSM, phenomenology, collider, Bsee
%    \item Lattice QCD -> Heavy quarks
%    \item Decay constants
%    \item RHQ, code implementation (also fermion flow)
%    \item GF renormalisation
%    \item Mixing, lifetimes on lattice + sum rules
%    \item $B_s\to D_s^*$
%    \item Quantum computing
%\end{itemize}
My research interests primarily lie around lattice QCD simulations in the context of particle physics phenomenology, with a focus on heavy quark physics.
Further interests are in QCD sum rules, BSM phenomenology, and quantum computing.

My academic research began during my Master's project, supervised by Prof.~Dr.~Alexander Lenz, where I delved into the Two Higgs Doublet model, and this work extended into the start of my PhD, where I performed comprehensive analyses using almost 200 observables to constrain 2HDMs.
The outcome of this research has resulted in three publications directly related to the 2HDM and its parameter space. 
Additionally, this work inspired further research into BSM phenomenology which lead to a fourth publication, exploring methods to enhance negligible decays to observable levels; this research invoked dialogue with experimental colleagues and further motivated LHCb researchers to look for such processes.
My involvement in BSM physics has significantly shaped the foundation of my research career, preparing me with vital skills which I apply in my ongoing work. 
It has also imparted me with a profound understanding of phenomenology at the LHC and other colliders, a knowledge base I consider essential to recognising physical meaning, context, and importance to my research.

My PhD commenced in March 2021, focusing on lattice QCD for heavy quark flavour physics under the mentorship of Dr.~Oliver Witzel. 
Initial research focused on the calculation of pseudoscalar and vector $B^{(*)}$, $B_s^{(*)}$, and $B_c^{(*)}$ meson decay constants as part of the RHQ project of the RBC/UKQCD collaboration. 
I also played a pivotal role in implementing the RHQ operators for $O(a)$ improvement in general leptonic and semileptonic decays within the C++ codebase {\tt Hadrons}. 
This also involved working with and advising a Bachelor's student, whose thesis work was involved with this code implementation.

This early research into heavy quark dynamics on the lattice using relativistic heavy quarks has continued throughout my PhD with interest in studying new types of decays on the lattice and the calculation of form factors for both pseudoscalar-to-pseudoscalar and pseudoscalar-to-vector transitions.

One of the primary research objectives of my PhD has been the development of techniques aimed at calculating the four-quark dimension-six $\Delta B=0$ matrix elements for $B$ meson lifetimes within lattice QCD. 
A key challenge for these matrix elements is the issue of operator mixing in standard renormalisation procedures. 
To address this, I have been working on a novel non-perturbative renormalisation scheme, leveraging the gradient flow and the ``short flow time expansion" in which operator mixing is absent; the preliminary work testing this method focuses on $\Delta F=2$ four-quark matrix elements governing neutral meson mixing such that we can compare to literature, where we first start at the charm quark scale before any additional extrapolations towards the bottom quark are to be considered.
With this innovative approach, we aim to provide full lattice QCD predictions of these vital matrix elements for the first time, working towards more accurate and precise predictions for the lifetimes of $B$ mesons.
Using knowledge gained from my work in the RHQ project, a first milestone for this research was also implementing the necessary code for the gradient flow in {\tt Hadrons}.\\
To perform the early simulations of this project, I made use of compute time on the large-scale High Performance Computing clusters {\tt HAWK} and {\tt LUMI-G}, where I have been in charge of compiling code, benchmarking, and running the simulations as well as the subsequent analysis. 
In particular on {\tt LUMI-G}, the use of GPU nodes made optimal compilation and benchmarking crucial before performing measurements.
Furthermore, for future use of {\tt LUMI-G} with larger-volume simulations, I was involved in writing a grant proposal for this needed compute time.

In parallel to my lattice research, I have also engaged in exploring alternative non-perturbative methodologies, namely QCD sum rules, for the computation of $\Delta B=0$ lifetime and $\Delta B=2$ mixing matrix elements similarly to my lattice studies, however here for BSM operators.
The QCD sum rules approach is quite different to lattice simulations, technically involving three-loop integrals, and this interdisciplinary approach has broadened my research horizon and enabled me to acquire a comprehensive understanding of diverse computational techniques in non-perturbative calculations.

\pagebreak
Outside of my core research, I have been a driving force in a quantum computing initiative within the Siegen physics community, bringing together the theoretical particle physics group with experimental quantum computing to learn the prospects of using quantum computing for particle physics. 
While our research experience in this field is still limited, I have actively pushed to learn more, as evidenced for example by my participation in the MIAPbP program ``Quantum Computing Methods for High Energy Physics".

Additionally, I have assumed significant responsibilities within the Sys Admin team of the TP1 group at Siegen, overseeing the management and maintenance of the computer systems. 
This multifaceted role involves not only the technical aspects of maintaining Linux systems but also encompasses the crucial responsibility of educating and assisting group members in navigating the intricacies of these systems and introducing new services to facilitate research endeavours.

In conclusion, I am eager to apply my extensive expertise and interdisciplinary knowledge to contribute to cutting-edge and innovative research in lattice QCD simulations and heavy quark physics, while also exploring novel computational methodologies. 
My research experience so far has covered all aspects of lattice QCD, such as code implementation, simulation running, and data analysis, and as such I am strongly suited to continue researching within this field with the flexibility and aptitude to perform in the many roles that can be required.
I am committed to leveraging my diverse skill set and comprehensive understanding to make significant contributions to the academic community and push the boundaries of our understanding of particle physics phenomenology.

\end{cvletter}

%-------------------------------------------------------------------------------
% Print the signature and enclosures with above letter informations
\makeletterclosing

\end{document}
