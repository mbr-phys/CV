%%%%%%%%%%%%%%%%%%%%%%%%%%%%%%%%%%%%%%%%%
% Freeman Curriculum Vitae
% XeLaTeX Template
% Version 2.0 (19/3/2018)
%
% This template originates from:
% http://www.LaTeXTemplates.com
%
% Authors:
% Vel (vel@LaTeXTemplates.com)
% Alessandro Plasmati
%
% License:
% CC BY-NC-SA 3.0 (http://creativecommons.org/licenses/by-nc-sa/3.0/)
%
%!TEX program = xelatex
% NOTICE: This template must be compiled with XeLaTeX, the line above should
% ensure this happens automatically but if it doesn't you will need to specify
% XeLaTeX as the engine in your editor or script
%
%%%%%%%%%%%%%%%%%%%%%%%%%%%%%%%%%%%%%%%%%

%----------------------------------------------------------------------------------------
%	PACKAGES AND OTHER DOCUMENT CONFIGURATIONS
%----------------------------------------------------------------------------------------

\documentclass[10pt]{article} % Font size, can be: 10pt, 11pt or 12pt

%%%%%%%%%%%%%%%%%%%%%%%%%%%%%%%%%%%%%%%%%
% Freeman Curriculum Vitae
% Structure Specification File
% Version 1.0 (19/3/2018)
%
% This template originates from:
% http://www.LaTeXTemplates.com
%
% Authors:
% Vel (vel@LaTeXTemplates.com)
% Alessandro Plasmati
%
% License:
% CC BY-NC-SA 3.0 (http://creativecommons.org/licenses/by-nc-sa/3.0/)
% 
%%%%%%%%%%%%%%%%%%%%%%%%%%%%%%%%%%%%%%%%%

%----------------------------------------------------------------------------------------
%	PACKAGES AND OTHER DOCUMENT CONFIGURATIONS
%----------------------------------------------------------------------------------------

\usepackage{etoolbox} % Required for conditional statements

\setlength\parindent{0pt} % Stop paragraph indentation

\usepackage{supertabular} % Required for the supertabular environment which allows tables to span multiple pages so sections with tables correctly wrap across pages

%----------------------------------------------------------------------------------------
%	DOCUMENT MARGINS
%----------------------------------------------------------------------------------------

\usepackage{geometry} % Required for adjusting page dimensions and margins

\geometry{
	hmargin=1.5cm, % Horizontal margin
	vmargin=1.75cm, % Vertical margin
	a4paper, % Paper size, change to letterpaper for US letter size
	%showframe, % Uncomment to show how the type block is set on the page -- typically for debugging
}

%----------------------------------------------------------------------------------------
%	COLUMN LAYOUT
%----------------------------------------------------------------------------------------

\usepackage{paracol} % Required for creating multi-column layouts that can span pages automatically

\columnratio{0.55,0.45} % The relative ratios of the two columns in the CV

\setlength\columnsep{0.05\textwidth} % Specify the amount of space between the columns

%----------------------------------------------------------------------------------------
%	FONTS
%----------------------------------------------------------------------------------------

\usepackage{fontspec} % Required for specifying custom fonts under XeLaTeX

\setmainfont{EBGaramond}[ % Make the default font EBGaramond
Path=fonts/, % The font is provided with the template in the fonts folder
UprightFont=*-Regular.ttf,
BoldFont=*-Bold.ttf,
BoldItalicFont=*-BoldItalic.ttf,
ItalicFont=*-Italic.ttf,
SmallCapsFont=*-SC.ttf
]

\newfontfamily\cvtextfont[Path=fonts/]{freebooterscript} % Create a new font family for the cursive font Freebooter Script, provided with the template in the fonts folder

\newfontfamily{\FA}[Path=fonts/]{FontAwesome} % Create a new font family for FontAwesome icons, provided with the template in the fonts folder
\input{fonts/fontawesomesymbols-xeluatex.tex} % Load a file to create shortcuts to the icons, see icon examples and their shortcuts in fontawesome.pdf in the fonts folder

\usepackage[sf,scale=0.95]{libertine} % Load Libertine as a \sffamily font for sans serif titles

%----------------------------------------------------------------------------------------
%	COLOURS AND LINKS
%----------------------------------------------------------------------------------------

\usepackage[usenames,svgnames]{xcolor} % Allows the definition and use of custom colours

\definecolor{text}{HTML}{2b2b2b} % Main document font colour, off-black
\definecolor{headings}{HTML}{701112} % Dark red colour for headings
\definecolor{shade}{HTML}{F5DD9D} % Peach colour for the contact information box
\definecolor{linkcolor}{HTML}{641c1d} % 25% desaturated headings colour for links
% Other colour options: shade=B9D7D9 and linkcolor=A40000; shade=D4D7FE and linkcolor=FF0080

% For preset colours that can be used by their names without having to define them, see: https://www.latextemplates.com/svgnames-colors

\color{text} % Set the default text colour for the whole document to the colour defined as 'text' above

%------------------------------------------------

\usepackage{hyperref} % Required for links

\hypersetup{colorlinks, breaklinks, urlcolor=linkcolor, linkcolor=linkcolor} % Set up links and their colours


%----------------------------------------------------------------------------------------
%	HEADERS & FOOTERS
%----------------------------------------------------------------------------------------

\usepackage{fancyhdr} % Required for customising headers and footers

\pagestyle{fancy} % Enable custom headers and footers

\fancyhf{} % This suppresses all headers and footers by default, add headers and footers in the template file as per the example

\renewcommand{\headrulewidth}{0pt} % Remove the default rule under the header

%----------------------------------------------------------------------------------------
%	SECTIONS
%----------------------------------------------------------------------------------------

\usepackage[nobottomtitles*]{titlesec} % Required for modifying sections, the nobottomtitles* is required for pushing section titles to the next page when they are close to the bottom of the page

\renewcommand{\bottomtitlespace}{0.1\textheight} % Modify the minimal space required from the bottom margin not to move the title to the next page

\titleformat{\section}{\color{headings}\scshape\LARGE\raggedright}{}{0em}{}[\color{black}\titlerule] % Define the \section format

\titlespacing{\section}{0pt}{0pt}{8pt} % Spacing around section titles, the order is: left, before and after

%----------------------------------------------------------------------------------------
%	CUSTOM COMMANDS
%----------------------------------------------------------------------------------------

% Command for entering a new work position
\newcommand{\workposition}[5]{
	 % Duration and conditional full time/part time text
	\expandafter\ifstrequal\expandafter{#3}{}{}{{\raggedright\large #3}} % Employer
    {\hfill{#1\expandafter\ifstrequal\expandafter{#2}{}{}{\hspace{6pt}\footnotesize{(#2)}}}\\}
	\expandafter\ifstrequal\expandafter{#4}{}{}{{\raggedright\large\textit{\textbf{#4}}}\\[4pt]} % Job title
	\expandafter\ifstrequal\expandafter{#5}{}{}{#5\\} % Description
}

% Command for entering a separate qualification
\newcommand{\educationentry}[5]{
	\textsc{#1} & \textbf{#2}\\ % Duration and degree
	\expandafter\ifstrequal\expandafter{#3}{}{}{& {\small\textsc{#3}}\\} % Honours, achievements or distinctions (e.g. first class honours)
	\expandafter\ifstrequal\expandafter{#4}{}{}{& #4\\} % Department
	\expandafter\ifstrequal\expandafter{#5}{}{}{& \textit{#5}\\[6pt]} % Institution
}

% Command for entering a separate table row -- used as a generic visual element for any section that uses a two column table
\newcommand{\tableentry}[3]{
	\textsc{#1} & #2\expandafter\ifstrequal\expandafter{#3}{}{\\}{\\[6pt]} % First the heading, then content, then a conditional insertion of whitespace if the third parameter has any content in it
}

% Command for entering a long-form description where there is a title on one line and a paragraph description below it
\newcommand{\longformdescription}[2]{
	\textit{#1}\\[3pt]
	#2\medskip
}

% Command for entering a publication in long-form format
\newcommand{\longformpublication}[1]{
	#1\medskip
}

% Command for entering a publication as a short DOI (digital object identifier) string to the publication, a link is automatically created
\newcommand{\doipublication}[4]{
	#1 & % Year
	\href{http://dx.doi.org/#2}{\expandafter\ifstrequal\expandafter{#3}{firstauthor}{\textbf{doi:#2}}{doi:#2}}% DOI string and if "firstauthor" is entered for the 3rd argument, this makes the DOI string bold indicating a first author publication
	\expandafter\ifstrequal\expandafter{#4}{}{\\}{\\[3pt]} % Conditional insertion of whitespace if the 4th parameter has any content in it
}
 % Include the file that specifies the document structure

% Headers and footers can be added with the \lhead{} \rhead{} \lfoot{} \rfoot{} commands
% Example right footer:
\rfoot{\color{headings}{\sffamily Last update: \today.}}
\usepackage{enumitem}
\usepackage{amsmath}
\usepackage{pifont}

%----------------------------------------------------------------------------------------

\begin{document}

\begin{paracol}{2} % Begin the multi-column environment

%----------------------------------------------------------------------------------------
%	NAME AND CURRICULUM VITAE TEXT
%----------------------------------------------------------------------------------------

\parbox[top][0.12\textheight][c]{\linewidth}{ % Parbox to hold the author name and CV text; fixed height to match the coloured box to the right, centred vertically and full line width
	\vspace{-0.04\textheight} % Reduce whitespace above the parbox to separate it from the main content
	\centering % Centre text
	{\sffamily\Huge Matthew Black}\\\medskip % Your name
	{\Huge\color{headings}\cvtextfont Curriculum Vitae}
}
%----------------------------------------------------------------------------------------
%	EDUCATION
%----------------------------------------------------------------------------------------

\section{Education}

% Blank \educationentry{} command to add another degree:

%\educationentry{} % Duration
%{} % Degree
%{} % Honours, achievements or distinctions (e.g. first class honours)
%{} % Department
%{} % Institution

% All 5 parameters must be supplied but any can be empty if you don't need them

%------------------------------------------------

\begin{supertabular}{rp{6.7cm}} % Start a table with two columns, the table will ensure everything is aligned

	%------------------------------------------------

	\educationentry{2016 -- 2020} % Duration
	{Master of Physics (MPhys)} % Degree
    {First Class Honours}
	{Theoretical Physics} % Department
	{University of Durham} % Institution

	%------------------------------------------------

	\educationentry{2009 -- 2016} % Duration
	{Secondary Education} % Degree
	{} % Honours, achievements or distinctions (e.g. first class honours)
    {} % Department
    {The High School of Glasgow} % Institution

	%------------------------------------------------

\end{supertabular}


%----------------------------------------------------------------------------------------
%	MAJOR RESEARCH PROJECT
%----------------------------------------------------------------------------------------

%\section{Masters Thesis}
%
%{\raggedright\textbf{``CP Violation In and Beyond The Standard Model: Two Higgs Doublet Model Type II Contributions to Flavour Observables"}\\\medskip}
%
%The Standard Model of particle physics is one of the great achievements of the 20$^{\text{th}}$ century. 
%This describes the fundamental interactions of particles and successfully unifies all observed forces of nature apart from gravity. 
%However, there are still many things that cannot be explained by the Standard Model, such as how the conditions of the early universe after the Big Bang lead to our current universe existing. 
%These questions require adapting and adding to the Standard Model to find answers which are also consistent with the rest of physics.
%This thesis studies a well-known extension of the Standard Model called the Two Higgs Doublet Model (2HDM), where four new Higgs bosons are introduced in addition to the famous particle found in 2012.
%This model has the potential to explain several of the problems of the Standard Model in the early universe, but is dependent on the specific values of the parameters of the 2HDM. 
%The 2HDM was used to modify the predictions of many important processes for many different values of two important parameters, and the new predictions were compared to experimental values for each process. 
%Upper and lower bounds on the values of the parameters of the 2HDM were then found, along with the statistical significance of these values. 
%These bounds inform us where we could confirm this model in experiment and still be consistent with other experiments.
%\\\\
%We first cover the theory of the Standard Model and then test the 2HDM of Type II as an extension to the Standard Model using indicative flavour observables, such as leptonic and semileptonic decays of $B$ and $D$ mesons, $B\bar{B}$ mixing, and the $b\to s\gamma$ radiative decay.
%Testing the 2HDM parameter space ($m_{H^+},\tan\beta$) to find alignment between theoretical calculations and experiment, constraints on the parameters were found for flavour phenomena to work towards a global fit.
%%We perform this fit in both the alignment and wrong sign limits of the 2HDM.
%Strongly dominated by the $b\to s\gamma$ branching ratio, the mass of a charged Higgs particle would be expected to have a minimum value at 95\% CL of $490$ GeV.% in the alignment limit and $740$ GeV in the wrong sign limit.
%The value of $\tan\beta$ is limited by $B_q\to\mu^+\mu^-$ decays, yielding a maximum value at 95\% CL of $20.8$.% in the alignment limit and $4.03$ in the wrong sign, for the fixed choices of other parameters.
%This work was then joined with constraints from the oblique parameters $S,T,U$ from another work, to better give judgement on the state of the 2HDM fit.
%The results of this combined fit are no more conclusive than flavour alone, due to additional parameter dependency in $S,T,U$; the only additional constraint we definitively find here is that $m_{H^+}\approx m_{H^0}\approx m_{A^0}$ in both the alignment and wrong sign limits.
%The validity of this fit is heavily influenced by the semileptonic ratio $\mathcal{R}(D^*)$ which remains in disagreement with the 2HDM fit to $3\sigma$; the statistical fit of these observables points to exclusion of this model at 95\% CL in the wrong sign limit, and 85\% CL in the alignment limit, at $2\sigma$ error significance.
%This model cannot be excluded at $3\sigma$ error significance, where $\mathcal{R}(D^*)$ no longer causes disagreement in the fit.

%\medskip % Extra whitespace before the next section

%------------------------------------------------

%\vspace{-\baselineskip}\medskip % Standardise the whitespace after this section and before the next (the custom command adds too much otherwise)

\section{Positions}

% Blank \workposition command to add another job:

%\workposition{} % Duration
%{} % FT/PT (full time or part time)
%{} % Employer
%{} % Job title
%{} % Description

% All 5 parameters must be supplied but any can be empty if you don't need them

%------------------------------------------------

\workposition{March 2021 -- present}
{FT}
{Universit\"{a}t Siegen}
{PhD Student}
{I am currently working as a PhD student in the TP1 research group for particle physics at Universit\"{a}t Siegen, focusing on lattice simulations of QCD. This role involves participating in research projects requiring strong dedication and focus. I work with others both within the TP1 group and internationally to complete research, and also as part of larger collaborations focused on delivering high precision lattice results. Further research interests include new physics models and quantum computation for high energy physics.}


%----------------------------------------------------------------------------------------


\section{Research Works}
See \href{https://inspirehep.net/authors/1886894?ui-citation-summary=true}{iNSPIRE-HEP/Matthew Black}.\\[8pt]
\longformpublication{\ding{229}\, {\bf M.~Black}, R.~Harlander, F.~Lange, A.~Rago, A.~Shindler and O.~Witzel, {\it Using Gradient Flow to Renormalise Matrix Elements for Meson Mixing and Lifetimes}, PoS LATTICE2023 XXX, \\ \phantom{i} [\href{https://arxiv.org/abs/2310.XXXX}{arXiv:2310.XXXX [hep-lat]}]}\\[5pt]
\longformpublication{\ding{229}\, {\bf M.~Black}, O. Witzel, {\it $B$ Meson Decay Constants Using Relativistic Heavy Quarks}, PoS LATTICE2022 405, [\href{https://arxiv.org/abs/2212.10125}{arXiv:2212.10125 [hep-lat]}]}\\[5pt]
\longformpublication{\ding{229}\, {\bf M.~Black}, A.~D.~Plascencia and G.~Tetlalmatzi-Xolocotzi, {\it Enhancing $B_s \to e^+ e^-$ to an Observable Level in the Two-Higgs-Doublet Model}, Phys.Rev.D 107 (2023) 3 035013, [\href{https://arxiv.org/abs/2208.08995}{arXiv:2208.08995 [hep-ph]}]}\\[5pt]
\longformpublication{\ding{229}\, O.~Atkinson, {\bf M.~Black}, C.~Englert, A.~Lenz and A.~Rusov, {\it MUonE, muon $g-2$ and electroweak precision constraints within 2HDMs}, \\ Phys.Rev.D 106 (2022) 11 115031, [\href{https://arxiv.org/abs/2207.02789}{arXiv:2207.02789 [hep-ph]}]}\\[5pt]
\longformpublication{\ding{229}\, O.~Atkinson, {\bf M.~Black}, C.~Englert, A.~Lenz, A.~Rusov and J.~Wynne, {\it The Flavourful Present and Future of 2HDMs at the Collider Energy Frontier}, JHEP 11 (2022) 139, [\href{https://arxiv.org/abs/2202.08807}{arXiv:2202.08807 [hep-ph]}]} \\[5pt]
\longformpublication{\ding{229}\, O.~Atkinson, {\bf M.~Black}, A.~Lenz, A.~Rusov and J.~Wynne, {\it Cornering the Two Higgs Doublet Model Type II}, JHEP 04 (2022) 172, \\ \phantom{i} [\href{https://arxiv.org/abs/2107.05650}{arXiv:2107.05650 [hep-ph]}]}

\switchcolumn % Switch to the next paracol column

%----------------------------------------------------------------------------------------
%	COLOURED CONTACT DETAILS BOX
%----------------------------------------------------------------------------------------

\parbox[top][0.12\textheight][c]{\linewidth}{ % Parbox to hold the colour box; fixed height to match the name/CV text to the left, centred vertically and full line width
	\vspace{-0.04\textheight} % Reduce whitespace above the parbox to separate it from the main content
	\colorbox{shade}{ % Create the coloured box
		\begin{supertabular}{p{0.05\linewidth}|p{0.795\linewidth}} % Start a table with two columns, the table will ensure everything is aligned
			%\raisebox{-1pt}{\faHome} & Easter Banknock, Denny, FK6 5NA \\ % Address
			%\raisebox{-1pt}{\faPhone} & +44 7446 949025, DoB: 26/10/1998 \\ % Phone number
            \raisebox{-1pt}{\faHome} & K\"{o}lner Str. 17, 57072 Siegen, Germany \\ % Address
			\raisebox{-1pt}{\faPhone} & +49 173 9173782, DoB: 26/10/1998 \\ % Phone number
            \raisebox{0pt}{\small\faEnvelope} & \raisebox{0pt}{\color{linkcolor}Matthew.Black@uni-siegen.de}\\%, \\ % Email address\href{mailto:Matthew.Black@uni-siegen.de}{Matthew.Black@uni-siegen.de}\\%, \\ % Email address 
            \raisebox{0pt}{\faGithub} & \raisebox{0pt}{\color{linkcolor}https://github.com/mbr-phys}\\ % GitHub profile\href{https://github.com/mbr-phys}{https://github.com/mbr-phys} \\ % GitHub profile
			%\raisebox{-1pt}{\small\faDesktop} & \href{https://www.LaTeXTemplates.com}{https://www.LaTeXTemplates.com} \\ % Website
			% See fontawesome.pdf in the fonts folder for all icons you can use
		\end{supertabular}
	}
}

%----------------------------------------------------------------------------------------
%	COMPUTER SKILLS
%----------------------------------------------------------------------------------------

\section{Computer Skills}

% Example \tableentry{} command to add another line:

%\tableentry{Heading}{Content}{spaceafter}

% All 3 parameters must be supplied but any can be empty if you don't need them
% A "spaceafter" value in the third parameter will add some vertical space -- this is to be used between headings

%------------------------------------------------

\begin{supertabular}{rl} % Start a table with two columns, the table will ensure everything is aligned

    \tableentry{Intermediate}{Fortran, Perl, HTML}{spaceafter}

	%------------------------------------------------

	\tableentry{Expert}{Python, C++, Unix, \LaTeX}{spaceafter}

	%------------------------------------------------

\end{supertabular}

\section{Further Experiences}
%
%% Example \longformdescription{} command to add another publication:
%
%%\longformpublication{Reference (format this manually as desired)}
%
%%------------------------------------------------
%
\longformdescription{Teaching Assistance}{Throughout my Masters and PhD, I have taken up teaching assistant duties for various courses. These include `Scientific Programming' and `Practical Lab: Intro to Lattice QCD', as well as advising a Bachelors student in their thesis. I have lent my knowledge of particle physics and programming in languages such as C++ and Python to good use in order to educate students and advance their own academic careers.}

\longformdescription{Systems Administrator}{During my time at UniSiegen, I have taken on responsibilities as part of the Sys Admin team managing and maintaining the computer systems of the TP1 group. This involves educating users on working with Linux systems and providing assistance and new services as needed by the group.}

%----------------------------------------------------------------------------------------
%	REFERENCES
%----------------------------------------------------------------------------------------

\section{References}

%\textit{References available on request}

%------------------------------------------------

% Example \tableentry{} command to add another line:

%\tableentry{Heading}{Content}{spaceafter}

% All 3 parameters must be supplied but any can be empty if you don't need them
% A "spaceafter" value in the third parameter will add some vertical space -- this is to be used between headings

%------------------------------------------------

\begin{supertabular}{rl} % Start a table with two columns, the table will ensure everything is aligned

	%------------------------------------------------

	\tableentry{}{\textbf{Dr. Alexander Lenz}}{spaceafter}
	\tableentry{Position}{Chair, Professor}{}
    \tableentry{Group}{{\color{linkcolor}\textit{Theoretical Physics 1}}}{}
    \tableentry{}{{\color{linkcolor}\textit{Universit\"{a}t Siegen}}}{spaceafter}
    \tableentry{Email}{{\color{linkcolor}Alexander.Lenz@uni-siegen.de}}{}
    \tableentry{Phone}{+49 (0)271/740-3890}{}

    %------------------------------------------------

	\tableentry{}{}{} % Creates some additional whitespace between the references

	%------------------------------------------------

    \tableentry{}{\textbf{Dr. Oliver Witzel}}{spaceafter}
    \tableentry{Position}{Akademischer Rat}{}
    \tableentry{Group}{{\color{linkcolor}\textit{Theoretical Physics 1}}}{}
    \tableentry{}{{\color{linkcolor}\textit{Universit\"{a}t Siegen}}}{spaceafter}
	\tableentry{Email}{{\color{linkcolor}Oliver.Witzel@uni-siegen.de}}{}
    \tableentry{Phone}{+49 (0)271/740-3703}{}

    %------------------------------------------------

	\tableentry{}{}{} % Creates some additional whitespace between the references

	%------------------------------------------------

    \tableentry{}{\textbf{Dr. Robert Harlander}}{spaceafter}
    \tableentry{Position}{Professor}{}
    \tableentry{Group}{{\color{linkcolor}\textit{Institute for Theoretical Particle Physics}}}{}
    \tableentry{}{{\color{linkcolor}\textit{and Cosmology}}}{}
    \tableentry{}{{\color{linkcolor}\textit{RWTH Aachen University}}}{spaceafter}
	\tableentry{Email}{{\color{linkcolor}harlander@physik.rwth-aachen.de}}{}
    \tableentry{Phone}{+49 (0)241/80-27045}{}

	%------------------------------------------------

%	\tableentry{}{}{} % Creates some additional whitespace between the references
%
%	%------------------------------------------------
%	\tableentry{}{\textbf{Mustafa Gun}}{spaceafter}
%	\tableentry{Position}{Food and Beverages Services Manager}{}
%    \tableentry{Employer}{\textit{St Mary's College}}{}
%	\tableentry{}{\href{https://www.dur.ac.uk}{\textit{University of Durham}}}{spaceafter}
%    \tableentry{Email}{\href{mailto:mustafa.gun@durham.ac.uk}{mustafa.gun@durham.ac.uk}}{}
%    \tableentry{Phone}{+44 (0)191 3345920}{}
%	\tableentry{Mobile}{+1 (232) 842-3583}{}


	%------------------------------------------------

\end{supertabular}
\end{paracol}

%----------------------------------------------------------------------------------------

\end{document}
